\documentclass[12pt]{john_resume}
\usepackage{kantlipsum}
\usepackage{relsize}
\usepackage{hyperref}
\hypersetup{
    colorlinks=true,
    urlcolor=blue
}

\begin{document}

\name{John Compitello}
\centerline{\url{https://github.com/johnc1231}}
\centerline{(631) 697-8777 $|$ compitello.j@husky.neu.edu $|$ linkedin.com/in/john-compitello}
\centerline{460 Parker Street, Boston, MA (School) $|$ 9 Brilner Drive, Smithtown, NY (Home)}

\newcommand{\worksection}[4]{\textlarger[1]{\textbf{#1}, #2 \hspace*{\fill}  #3 \\ \textit{#4}} }

\section{Education}
\datedsubsection{Northeastern University}{September 2015 - Present}

\section{Programming Experience}
\begin{tabular}{l l}
\textbf{Languages:} & Scala, Java, Python, SQL, C\\
\textbf{Frameworks:} & Apache Spark, JUnit \\
\textbf{Open Source Contributions:} & Apache Spark, Hail
\end{tabular}

\section{Work Experience}

\worksection{Broad Institute of MIT and Harvard}{Boston MA}{January - August 2017}{Software Engineering Intern on Hail team}
\begin{itemize}
	\item Translated statistical genetics methods into distributed algorithms in Apache Spark to analyze 
		terabytes of genetic data.
	\item Increased speed of linear mixed regression by 3x and maximum dataset size by 10x by utilizing 
		linear algebra theory to eliminate unnecessary computation, breaking up method into discrete parts 	
		that can be used separately, and rewriting inefficient Spark library methods. 
	\item Implemented method to generate synthetic genetic datasets to aid in testing and method 
		development.
\end{itemize}




\end{document}